\documentclass[a4paper]{article} 	% use "amsart" instead of "article" for AMSLaTeX format
\usepackage{geometry}  		% See geometry.pdf to learn the layout options. There are lots.
\geometry{left=1.5cm,right=1.5cm,top=1.5cm,bottom=1.5cm}   		
\usepackage{graphicx,subcaption}					
\usepackage{amssymb}
\usepackage{indentfirst}
\usepackage{amsmath}
\usepackage{amsthm}
\usepackage{bm}
\usepackage{lineno}
\usepackage{setspace}
\usepackage{booktabs,multirow}
\usepackage{authblk}
%\usepackage{subcaption}
\usepackage{graphicx}
\usepackage{float}
\usepackage[flushleft]{threeparttable} % For adding footnotes to tables


\RequirePackage[colorlinks,citecolor=blue,urlcolor=blue]{hyperref}

\newtheorem{theorem}{Theorem}
\newtheorem{lemma}[theorem]{Lemma}
\newtheorem{corollary}[theorem]{Corollary}
\newcommand{\E}{\mathrm{E}}
\newcommand{\D}{\mathcal{MD}}
\newcommand{\Var}{\mathrm{Var}}
\newcommand{\Cov}{\mathrm{Cov}}
\newcommand{\Corr}{\mathrm{Corr}}
\newcommand{\tr}{\mathrm{tr}}
\newcommand{\R}{\texttt{R}}
\newcommand{\asreml}{\texttt{ASReml-R}}
\newcommand{\Matern}{Mat\'ern }
\newcommand{\N}{\mathcal{N}}
\newcommand{\AR}{\mathrm{AR1}}
\newcommand{\BigO}[1]{{\rm O}\left(#1\right)}
\newcommand{\eg}{e.g.\ }
\newcommand{\ie}{i.e.\ }
\newcommand{\iid}{\textrm{i.i.d.\ }}

\newcommand{\revision}[1]{\textcolor{red}{#1}}


\usepackage[url=false, isbn=false, eprint=false, backref=true, style= authoryear, backend=biber, maxcitenames=2, giveninits=true, maxbibnames=100, uniquename=init]{biblatex}   %% backend=bibtex
\DeclareNameAlias{sortname}{family-given}
\addbibresource{REF.bib}

\title{Title of this manuscript}
%\author{Jerome}
\author[1]{First Author Name \thanks{Corresponding author: email@example.com}}
\author[1,2]{Second Author}
\author[2]{Third Author}



\affil[1]{Affiliation one}
\affil[2]{Affiliation two}
 
\date{}							
% Activate to display a given date or no date

\linenumbers
\doublespacing
%\onehalfspacing

\begin{document}

\maketitle


\begin{abstract}
The abstract would summarise the study's purpose, methods, results, and conclusions, such as how varying temperatures affect enzyme activity in a lab setting.
\end{abstract}

\section{Introduction}
The introduction would provide background information on enzyme activity and temperature sensitivity, leading to the hypothesis that temperature changes will affect enzyme activity. 

For citations, normal \textcite{einstein1905, Cressie1999Classes}, and with parentheses \parencite{Abramowitz1974Handbook}.

\section{Methods}
The methods section would detail how the experiment was conducted, including the enzymes studied, the temperature ranges tested, and the assay methods used to measure enzyme activity.

The Pythagorean theorem can be expressed as:
\begin{equation}\label{eq:example}
    a^2 + b^2 = c^2
\end{equation}
where $a$ and $b$ are the lengths of the legs of a right triangle, and $c$ is the length of the hypotenuse.

In the above equation \eqref{eq:example}, we learnt something. 

\begin{theorem}\label{th:example}
This is a theorem.
\end{theorem}

This is Theorem \ref{th:example}. 



\section{Results}
The results section would present the data collected, such as enzyme activity levels at different temperatures, possibly including tables or graphs to illustrate the findings.


Refer to Figure \ref{fig:single_figure}. 

\begin{figure}[H]
    \centering
    \includegraphics[width=0.5\textwidth]{example-image-a}
    \caption{Single Figure Caption}
    \label{fig:single_figure}
\end{figure}


Refer to Figures \ref{fig:subfig_a} and \ref{fig:subfig_b}. 

\begin{figure}[H]
    \centering
    \begin{minipage}{0.45\textwidth}
        \centering
        \includegraphics[width=\textwidth]{example-image-a}
        \subcaption{Sub-caption for Figure A}
        \label{fig:subfig_a}
    \end{minipage}%
    \hfill
    \begin{minipage}{0.45\textwidth}
        \centering
        \includegraphics[width=\textwidth]{example-image-b}
        \subcaption{Sub-caption for Figure B}
        \label{fig:subfig_b}
    \end{minipage}
    \caption{Two Figures Side by Side}
    \label{fig:two_figures}
\end{figure}




Refer to Table \ref{tab:my_table}. 

\begin{table}[H] %%% put table here "H", or "htbp"
    \centering
    \begin{tabular}{ccc}
        \toprule
        \textbf{Column 1} & \textbf{Column 2} & \textbf{Column 3} \\ \midrule
        Row 1, Col 1 & Row 1, Col 2 & Row 1, Col 3 \\
        Row 2, Col 1 & Row 2, Col 2 & Row 2, Col 3 \\
        Row 3, Col 1 & Row 3, Col 2 & Row 3, Col 3 \\
        \bottomrule
    \end{tabular}
    \caption{Your Table Caption Here}
    \label{tab:my_table}
\end{table}


\section{Discussion}
The discussion would interpret the results, comparing them to prior research and explaining the implications of the findings for understanding enzyme function and temperature sensitivity.

\section{Conclusion}
The conclusion would summarise the key findings and their significance, potentially suggesting avenues for future research or practical applications in fields like agriculture or medicine.

\section*{Acknowledgements}
Acknowledgements might include funding sources, lab assistants, or equipment providers.


\appendix



\renewcommand\bibname{References}% change bibliography title to references
	%\addcontentsline{toc}{chapter}{Bibliography}
\addtocontents{toc}{Bibliography}
\printbibliography
\end{document}
